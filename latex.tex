\documentclass[journal,12pt,twocolumn]{IEEEtran}

\usepackage{setspace}
\usepackage{gensymb}

\singlespacing


\usepackage[cmex10]{amsmath}

\usepackage{amsthm}

\usepackage{mathrsfs}
\usepackage{txfonts}
\usepackage{stfloats}
\usepackage{bm}
\usepackage{cite}
\usepackage{cases}
\usepackage{subfig}

\usepackage{longtable}
\usepackage{multirow}

\usepackage{enumitem}
\usepackage{mathtools}
\usepackage{steinmetz}
\usepackage{tikz}
\usepackage{circuitikz}
\usepackage{verbatim}
\usepackage{tfrupee}
\usepackage[breaklinks=true]{hyperref}
\usepackage{graphicx}
\usepackage{tkz-euclide}

\usetikzlibrary{calc,math}
\usepackage{listings}
    \usepackage{color}                                            %%
    \usepackage{array}                                            %%
    \usepackage{longtable}                                        %%
    \usepackage{calc}                                             %%
    \usepackage{multirow}                                         %%
    \usepackage{hhline}                                           %%
    \usepackage{ifthen}                                           %%
    \usepackage{lscape}     
\usepackage{multicol}
\usepackage{chngcntr}

\DeclareMathOperator*{\Res}{Res}

\renewcommand\thesection{\arabic{section}}
\renewcommand\thesubsection{\thesection.\arabic{subsection}}
\renewcommand\thesubsubsection{\thesubsection.\arabic{subsubsection}}

\renewcommand\thesectiondis{\arabic{section}}
\renewcommand\thesubsectiondis{\thesectiondis.\arabic{subsection}}
\renewcommand\thesubsubsectiondis{\thesubsectiondis.\arabic{subsubsection}}


\hyphenation{op-tical net-works semi-conduc-tor}
\def\inputGnumericTable{}                                 %%

\lstset{
%language=C,
frame=single, 
breaklines=true,
columns=fullflexible
}
\begin{document}


\newtheorem{theorem}{Theorem}[section]
\newtheorem{problem}{Problem}
\newtheorem{proposition}{Proposition}[section]
\newtheorem{lemma}{Lemma}[section]
\newtheorem{corollary}[theorem]{Corollary}
\newtheorem{example}{Example}[section]
\newtheorem{definition}[problem]{Definition}

\newcommand{\BEQA}{\begin{eqnarray}}
\newcommand{\EEQA}{\end{eqnarray}}
\newcommand{\define}{\stackrel{\triangle}{=}}
\bibliographystyle{IEEEtran}

\providecommand{\mbf}{\mathbf}
\providecommand{\pr}[1]{\ensuremath{\Pr\left(#1\right)}}
\providecommand{\qfunc}[1]{\ensuremath{Q\left(#1\right)}}
\providecommand{\sbrak}[1]{\ensuremath{{}\left[#1\right]}}
\providecommand{\lsbrak}[1]{\ensuremath{{}\left[#1\right.}}
\providecommand{\rsbrak}[1]{\ensuremath{{}\left.#1\right]}}
\providecommand{\brak}[1]{\ensuremath{\left(#1\right)}}
\providecommand{\lbrak}[1]{\ensuremath{\left(#1\right.}}
\providecommand{\rbrak}[1]{\ensuremath{\left.#1\right)}}
\providecommand{\cbrak}[1]{\ensuremath{\left\{#1\right\}}}
\providecommand{\lcbrak}[1]{\ensuremath{\left\{#1\right.}}
\providecommand{\rcbrak}[1]{\ensuremath{\left.#1\right\}}}
\theoremstyle{remark}
\newtheorem{rem}{Remark}
\newcommand{\sgn}{\mathop{\mathrm{sgn}}}
\providecommand{\abs}[1]{\left\vert#1\right\vert}
\providecommand{\res}[1]{\Res\displaylimits_{#1}} 
\providecommand{\norm}[1]{\left\lVert#1\right\rVert}
%\providecommand{\norm}[1]{\lVert#1\rVert}
\providecommand{\mtx}[1]{\mathbf{#1}}
\providecommand{\mean}[1]{E\left[ #1 \right]}
\providecommand{\fourier}{\overset{\mathcal{F}}{ \rightleftharpoons}}
%\providecommand{\hilbert}{\overset{\mathcal{H}}{ \rightleftharpoons}}
\providecommand{\system}{\overset{\mathcal{H}}{ \longleftrightarrow}}
	%\newcommand{\solution}[2]{\textbf{Solution:}{#1}}
\newcommand{\solution}{\noindent \textbf{Solution: }}
\newcommand{\cosec}{\,\text{cosec}\,}
\providecommand{\dec}[2]{\ensuremath{\overset{#1}{\underset{#2}{\gtrless}}}}
\newcommand{\myvec}[1]{\ensuremath{\begin{pmatrix}#1\end{pmatrix}}}
\newcommand{\mydet}[1]{\ensuremath{\begin{vmatrix}#1\end{vmatrix}}}

\numberwithin{equation}{subsection}

\makeatletter
\@addtoreset{figure}{problem}
\makeatother
\let\StandardTheFigure\thefigure
\let\vec\mathbf

\renewcommand{\thefigure}{\theproblem}

\def\putbox#1#2#3{\makebox[0in][l]{\makebox[#1][l]{}\raisebox{\baselineskip}[0in][0in]{\raisebox{#2}[0in][0in]{#3}}}}
     \def\rightbox#1{\makebox[0in][r]{#1}}
     \def\centbox#1{\makebox[0in]{#1}}
     \def\topbox#1{\raisebox{-\baselineskip}[0in][0in]{#1}}
     \def\midbox#1{\raisebox{-0.5\baselineskip}[0in][0in]{#1}}
\vspace{3cm}
\title{Assignment 17}
\author{KUSUMA PRIYA\\EE20MTECH11007}

\maketitle
\newpage

\bigskip
\renewcommand{\thefigure}{\theenumi}
\renewcommand{\thetable}{\theenumi}
Download codes from 
%
\begin{lstlisting}
https://github.com/KUSUMAPRIYAPULAVARTY/assignment17
\end{lstlisting}
%
 
 \section{QUESTION}
Let $\mathbb{F}$ be a field of characteristic zero and let $\vec{V}$ be a finite dimensional vector space over  $\mathbb{F}$.If $\alpha_1,\alpha_2,\hdots,\alpha_m$ are finitely many vectors in $\vec{V}$ , each different from the zero vector, prove that there is a linear functional $f$ on $\vec{V}$ such that
\begin{align}
    f(\alpha_i) \neq 0, i=1,2,\hdots,m
\end{align}
%

\section{Theorem}
We will make use of the following theorem\\
Let $\vec{V}$ be a finite dimensional vector space over the field $\mathbb{F}$ and let $\cbrak{\vec{v_1},\vec{v_2},\hdots,\vec{v_n}}$ be a basis for $\vec{V}$.Then there exists a unique dual basis $\cbrak{f_1,f_2,\hdots,f_n}$ for $\vec{V}^*$ such that
\begin{align}
    f_i(\vec{v}_j)=\delta_{ij}\label{1}
\end{align}
and $\vec{V}^*$ is the space of all linear functionals on $\vec{V}$
\section{Solution}
\begin{table*}[!ht]
\centering
\begin{tabular}{|c|c|}
\hline
\textbf{PARAMETERS}&\textbf{DESCRIPTION}\\
\hline
$\mathbb{F}$&Field\\
\hline
$\vec{V}$& Finite dimensional vector space over $\mathbb{F}$\\
\hline
$\alpha_1,\alpha_2,\hdots,\alpha_m$& non zero vectors in $\vec{V}$\\
\hline
$f:\vec{V}\rightarrow \mathbb{F}$&Linear functional on $\vec{V}$\\
\hline
\end{tabular}
\caption*{TABLE 0:Input Parameters}
\end{table*}

\begin{table*}[!hb]
\centering
\begin{tabular}{|c|c|}
\hline
\textbf{PARAMETERS}&\textbf{MATRIX REPRESENTATION}\\
\hline
 Basis for $\vec{V}$&$ \vec{B}=\myvec{\vec{v}_1&\vec{v}_2&\hdots&\vec{v}_n}$\\
\hline
Basis for $\vec{V}^*$& $\vec{B}^*=\myvec{\vec{f}_1&\vec{f}_2&\hdots&\vec{f}_n}$\\
\hline
 & \\
$f_i(\vec{v}_j)=\delta_{ij}$& $(\vec{f}_i)^T\vec{v}_j=\delta_{ij}$\\
\hline
 & \\
$\vec{B},\vec{B}^*$ are dual basis&$(\vec{B}^*)^T\vec{B}=\vec{I}$\\
\hline
$\alpha_i=\sum _{k=1}^n a_k\vec{v}_k,i\in [1,m]$&$ \alpha_i=\vec{B}\vec{a}$ where $ \vec{a}=\myvec{a_1\\a_2\\\vdots\\a_n}, a_l \in \mathbb{F},l \in [1,n]$$\\
\hline
$\alpha_i\neq \vec{0}$&$\vec{a}\neq\vec{0}$\\
\hline
Any linear functional $f$ over $\vec{V}$& $\vec{f}=\vec{B}^*\vec{c},\text{where }\vec{c}=\myvec{c_1\\c_2\\\vdots\\c_n}, c_l \in \mathbb{F} ,l \in [1,n]$\\
\hline
 & \\
 & $ \vec{f}^T\alpha_i$\\
$ f(\alpha_i)$ &$
    =(\vec{B}^*\vec{c})^T\alpha_i$\\
     &$=\vec{c}^T(\vec{B}^*)^T\vec{B}\vec{a}$\\
     &$=\vec{c}^T\vec{a}$ \\
\hline
 & \\
$f$ with  $c_l \neq 0 \forall l \in [1,n]$ lies in $\vec{V}^*,\vec{a}\neq\vec{0}$ & $f(\alpha_i)=\vec{c}^T\vec{a}\neq 0, i=1,2,\hdots,m $\\
\hline
\end{tabular}
\caption*{TABLE 1:Proof}
\end{table*}
\end{document}


